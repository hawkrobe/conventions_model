\subsection{Communities}

How do community-level conventions emerge from local interactions? 
In this paper, we proposed a partial-pooling account, formalized as a hierarchical Bayesian model, where conventions represent the shared structure that agents "abstract away" from partner-specific interactions.
Unlike complete-pooling accounts, this model allows for partner-specific common ground to override community-wide expectations given sufficient experience with a partner, or in the absence of strong conventions.
Unlike no-pooling accounts, it allows networks to converge on more efficient and accurate expectations about novel partners.
We conducted a series of simulations demonstrating the model's generalization behavior, and evaluated these predictions with a natural-language communication experiment on a small network.

Hierarchical Bayesian models have several other properties of theoretical interest for convention formation.
First, they offer a "blessing of abstraction" \cite{GoodmanUllmanTenenbaum11_TheoryOfCausality}, where community-level conventions may be learned even with relatively sparse input from each partner, as long as there is not substantial variance in the population. 
Second, they are more robust to partner-specific deviations from conventions (e.g. interactions with children or non-native speakers) than complete-pooling models relying on a fixed set of memory slots or a single mental 'inventory.' 
This robustness is due to their ability to 'explain away' outliers without community-level expectations being affected. 
Finally, the deep connection between hierarchical Bayesian models and accounts of *meta-learning*, or learning to learn \cite{grant_recasting_2018}, provides a useful set of tools to analyze conventions as the result of agents solving a meta-learning problem, adapting to each partner along the way.

Real-world communities are much more complex than the simple networks we considered: each speaker takes part in a number of overlapping subcommunities. 
For example, we use partially distinct conventions depending on whether we are communicating with psychologists, friends from high school, bilinguals, or children.
For future work using hierarchical Bayesian models to address the full scale of an individual's  network of communities, additional social knowledge about these communities must be learned and represented in the generative model \cite[e.g.]{gershman_learning_2017} 
Our results are a promising first step, providing evidence that hierarchical generalization may be a foundational cognitive building block for establishing conventionality at the group level.
