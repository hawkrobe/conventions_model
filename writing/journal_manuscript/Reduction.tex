
%If our lexical priors -- our global conventions -- serve as a source of stability in meaning over longer timescales, then what accounts for our extraordinary flexibility  over short timescales? How do we coordinate on efficient local conventions, or \emph{conceptual pacts}, for talking about things we've never talked about before? In this section, we review the dynamics of coordination within repeated reference games and explore the possibility, formalized in Chapter 2, that rapid adaptation can be understood in a Bayesian modeling framework as lexical inference given partner-specific data.%: $P(\mathcal{L}_i | D_i, \Theta)$. 

We begin with the phenomenon of increasing efficiency in repeated reference games: speakers use detailed descriptions at the outset but converge to increasingly compressed shorthand while remaining understandable to their partner.
While this phenomenon has been extensively documented, to the point of serving as a proxy for common ground \cite{}, it has continued to pose a challenge for computational models of language use.
One simple explanation --- that it is merely an effect of familiarity or repetition on the part of the speaker, not related to \emph{ad hoc} conventions --- can be easily dispelled. 
When participants are asked to repeatedly refer to the same targets for a \emph{hypothetical} partner, no decrease in utterance length is found; in some cases utterances actually get longer \cite{HupetChantraine92_CollaborationOrRepitition}. 
This control experiment suggests that whatever is changing must be a result of the \emph{interaction} between partners.

It is also not clear how increasing efficiency could be explained by the lower-level alignment mechanisms proposed in \emph{interactive alignment} accounts \cite{pickering2004toward, pickering2006alignment, garrod2009joint}.
According to these accounts, coordination on meaning proceeds primarily through automatic priming of surface statistics at lower levels.
While low-level priming is certainly possible in repeated reference tasks, especially when listeners engage in extensive dialogue, it is not clear why priming would favor some words in a long referring expression over others, or lead to words being dropped.
Furthermore, priming alone cannot explain why speakers still converge to more efficient labels even when the listener is prevented from saying anything at all and only indirect feedback about the listener's accuracy is provided \cite{KraussWeinheimer66_Tangrams} and keep using longer descriptions given evidence of errors \cite{hawkins2020characterizing}.
Explaining why speakers increasingly think that shorter descriptions will suffice requires a mechanism for coordination on meaning even given sparser, non-verbal feedback.

Finally, we consider agent-based models implementing simple update rules \cite{steels_self-organizing_1995,barr_establishing_2004,young_evolution_2015}.
These models all share some mechanism that makes utterances more likely to be produced after communicative successes and less likely after communicative failures.
It is not clear why agents using such rules would initially prefer to produce longer utterance without some notion of uncertainty, or how naively reinforcing initially long descriptions could lead to reduction.% without some mechanism for \emph{credit assignment} to the component words.
When reduction has been investigated in these models (e.g. phonological erosion), the process has been hard-coded as an $\epsilon$ probability of speakers dropping a token, which listeners are able to use to recover the un-truncated form \cite{beuls2013agent,steels2016agent}.
\todo[inline]{need a transition here}

An alternative explanation suggested by our model is that efficiency is driven by \emph{ad hoc} word learning as communication partners make more observations of one another's usage. 
Long initial messages may be explained as the result of initial uncertainty in the speaker's lexical prior.
Then shorter, more efficient messages may be licensed as the speaker grows more confident about the listener's understanding, via feedback.

\subsection{Model simulations}

\begin{figure*}
\centering
    \includegraphics[scale=.9]{sec1-modelResults.pdf}
  \caption{Schematic of model}
  \label{fig:sec1model}
\end{figure*}


In this section, we show that the mechanism of uncertainty reduction is sufficient to produce reduction in utterance length across repeated interaction.
\todo[inline]{Give a short version of the explanation here.}
\paragraph{Simulation 1.1: Coordination}

First, we show how agents updating their meaning functions in this way can coordinate even in the absence of strong initial priors. 
The initial choices in an interaction can be taken as evidence for a particular lexicon and become the basis for successful communication, even when both speaker and listener are uncertain at the beginning.
As a simple test case, consider an environment with two objects ($\{o_1, o_2\}$), where the speaker must choose between two utterances ($\{u_1, u_2\}$) with equal production costs. 

\todo[inline]{unpack what $\phi_k$ means, what some examples of functions are, etc.}
To define the prior $P(\phi_k)$ over the meaning of each utterance, we must first say what representation we are using for the lexicon $\mathcal{L}_{\phi_k}$.
For simplicity, we follow prior Bayesian models of word learning \cite{XuTenenbaum07_WordLearningBayesian} and represent the space of meanings as the set of possible sub-trees in a concept taxonomy.
When objects are conceptually unrelated, as in most prior work on signaling games, this taxonomy is flat, and the space of meanings is equivalent to the set of objects.
Further, we assume a simplicity prior over possible lexicons, $P(\phi) \propto \exp\{|\phi|\}$, where $|\phi|$ is the total size of each word's extension, summed across words in the vocabulary \cite{FrankGoodmanTenenbaum09_Wurwur}.
\footnote{There are many alternative representational choices compatible with our core model, including parameterized vector embeddings and multi-layer neural networks, which may be more appropriate for scaling our model to larger spaces of words and referents. We return to these possibilities in the General Discussion.}
On the first round both utterances are equally likely to apply to either shape. 

If the speaker were trying to get their partner to pick $o_1$, then because each utterance is equally (un)informative, they could only randomly sample one (say, $u_1$), and observe the listener's selection of a shape (say, $o_1$, a correct response). 
On the next round, the speaker uses the observed pair $\{u_1, o_1\}$ to update their beliefs about their partner's true lexicon, uses these beliefs to generate a new utterance, and so on. 
To examine expected dynamics over multiple rounds, we forward sample many possible trajectories.

We observe several important qualitative effects in our simulations. 
First, and more fundamentally, the evidence that a knowledgeable listener responded to utterance $u$ by choosing a particular object $o$ provides support for lexicons in which $u$ is a good fit for $o$. 
Hence, the likelihood of the speaker using $u$ to refer to $o$ will increase on subsequent rounds (see Fig.\ref{fig:sec1model}A). 
In other words, the initial symmetry between the meanings can be broken by initial random choices, leading to completely arbitrary but stable mappings in future rounds. 

Second, because the listener is updating their meaning representation from the same observations under the same set of assumptions, both partners converge on a \emph{shared} set of meanings; hence, the expected accuracy of selecting the target object rises on future rounds (see Fig. \ref{fig:sec1model}B). 
Third, because one's partner is assumed to be pragmatic via recursive Rational Speech Act mechanisms, agents can also learn about \emph{unheard} utterances. 
Observing $d = \{(u_1, o_1)\}$ also provides evidence that $u_2$ is \emph{not} a good fit for $o_1$.
This effect arises from Gricean maxims: if $u_2$ were a better fit for $o_1$, the speaker would have used it instead \cite{Grice75_LogicConversation}. 
Fourth, \emph{failed references} can lead to conventions just as effectively as successful references: if the speaker intends $o_1$ and says $u_1$, but then the listener incorrectly picks $o_2$, the speaker will take this as evidence that $u_1$ actually means $o_2$ in their partner's lexicon and become increasingly likely to use it that way on subsequent rounds.

\paragraph{Simulation 1.2: Reduction}

Next, we show how our model explains reduction of utterance length over multiple interactions. 
\todo[inline]{Splinter off panel C to show illustration of how this works}
For utterances to be reduced, of course, they must vary in length, so we extend our grammar to include \emph{conjunctions}. 
Conjunctions are one of the simplest ways to construct longer, non-atomic utterances compositionally from lexical primitives, using the product rule \cite<see also>[who instead consider negation]{SteinertThrelkeld16_CompositionalSignaling}:
$$\mathcal{L}(u_i \textrm{ and } u_j, o) = \mathcal{L}(u_i, o) \times \mathcal{L}(u_j, o)$$
We consider a scenario where two objects $\{o_1, o_2\}$ differ along two different features. 
The speaker thus has four primitive words at their disposal -- two words for the first feature ($\{u_{11}, u_{12}\}$) and two for the second $\{u_{21}, u_{22}\}$. 
While we established in the previous section that conventions can emerge over a reference game in the complete absence of initial preferences, players often bring such preferences to the table. 
A player who hears `ice skater' on the first round of a tangrams task is more likely to select some objects more than others, even though they still have some uncertainty over its meaning in the context. 
To show that our model can accommodate this fact, we allow the speaker's initial prior meanings to be slightly biased. 
We assume $u_{11}$ and $u_{21}$ are a priori more likely to mean $o_1$ and $u_{12}$ and $u_{22}$ are more likely to mean $o_2$.

We ran 1000 forward samples of 6 rounds of speaker-listener interaction, and averaged over the utterance length at each round \footnote{In our simulations, we used $\alpha = 10$ but found the basic reduction effect over a range of different biases}. 
Our results are shown in Figure \ref{fig:sec1model}C: the expected utterance length decreases systematically over each round. 
To illustrate in more detail how this dynamic is driven by an initial rational preference for redundancy relaxing as reference becomes more reliable, we walk step-by-step through a single trajectory. 
Consider a speaker who wants to refer to object $o_1$. 
They believe their knowledgeable partner is slightly more likely to interpret their language using a lexicon in which $u_{11}$ and $u_{12}$ apply to this object, due to their initial bias. 
However, there is still a reasonable chance that one or the other alone actually refers more strongly to $o_2$ in the true lexicon. 
Thus, it is useful to produce the conjunction "$u_{11}$ and $u_{12}$" to hedge against this possibility, despite its higher cost. 
Upon observing the listener's response (say, $o_1$), the evidence is indeterminate about the separate meanings of $u_{11}$ and $u_{12}$ but both become increasingly likely to refer to $o_1$. 
In the trade-off between informativity and cost, the shorter utterances remain probable options. 
Once the speaker chooses one of them, the symmetry collapses and that utterance remains most probable in future rounds. 
In this way, meaningful sub-phrases are omitted over time as the speaker becomes more confident about the true lexicon. 

\paragraph{Model comparison}

Here we compare this model to several simpler baselines to establish which components of the model are necessary and sufficient for the desired behavior.

\rdh{e.g., no pragmatics, pragmatics only in learning rule or only in decision rule instead of both, simpler pragmatics (reasoning about $L_0$ instead of $L_1$), point estimate instead of uncertainty, effect of different parameter regimes.} 

\rdh{It may also be useful to explicitly show that the simpler Roth-Erev RL updating from this literature doesn't reduce, or even better show that this kind of simpler update rule is equivalent to something within our framework as a point estimate representation with maximum likelihood or something...?}

%In the limit, it doesn't matter whether you have pragmatics in both learning rule or decision rule. 
%In case where it's only in production rule, you'll produce the data with the necessary biases in learning.



\subsection{Discussion}

The explanation for increasing efficiency we have offered is consistent with some additional observations from prior empirical work.
For example, if participants reduce their lexical uncertainty over successive rounds, as we suggest, then we might expect a corresponding decrease in explicit markers of this uncertainty. 
Indeed, \citeA{BrennanClark96_ConceptualPactsConversation} counted \emph{hedges} across repetitions.
Hedges are expressions like \emph{sort of} or \emph{like}, and morphemes like \emph{-ish}, that explicitly mark uncertainty or provisionality, such as \emph{a car, sort of silvery purple colored} \cite{BrennanClark96_ConceptualPactsConversation,Fraser10_Hedging,MedlockBriscoe07_HedgeClassification}.
\citeA{BrennanClark96_ConceptualPactsConversation} found a much greater occurrence of hedges on the first round than the final round (26\% and 2\%, respectively).
Additionally, very few hedges were found on early trials in less ambiguous contexts (e.g. referring to a shoe in the context of dogs and fish), lending support for the specific use of hedges to mark uncertainty rather than a generic social use when first beginning to talk with a new partner.

\todo[inline]{Discuss specific relationships to other models, e.g. also the observation about }
While simple reinforcement learning models It is plausible that reinforcement learning models using more sophisticated algorithms could predict patterns of reduction with the addition of a cost term. 
For instance, neural network architectures appropriately incorporating compositionality and recurrence into production may be able to implicitly ground shorter utterances in prior usage using gradient-based learning \cite{hawkins2019continual}.
However, such a scheme would be much closer in spirit to our model.
We return to this issue in our discussion of the scalability of our model in the General Discussion.

Finally, our simulations in this section are consistent with recent analyses of exactly \emph{what} gets reduced \cite{hawkins2020characterizing}.
Is the speaker adopting a fragment shorthand by randomly and noisily dropping words, or are they simplifying or narrowing their descriptions to names by systematically omitting redundant details?
Closed-class parts of speech like determiners and prepositions \emph{are} much more likely to be dropped than open-class parts of speech like adjectives and nouns, and entire modifying clauses are more likely to be dropped together than expected by random corruption.
This accords with early hand-tagged analyses by \citeA{Carroll80_NamingHedges}, which found that in three-quarters of transcripts from \citeA{krauss_changes_1964} the short names that participants converged upon were prominent in some syntactic construction at the beginning, often as a head noun that was initially modified or qualified by other information. 
These more fine-grained analyses suggest that reduction is grounded in the prior lexical content of the interaction and the speaker's increasing confidence in how the listener will interpret an initially ambiguous label. 

