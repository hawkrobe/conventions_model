
% 4 paragraph intro

To communicate successfully, speakers and listeners must share a common system of semantic meaning in the language they are using. 
These meanings are \emph{social conventions} in the sense that they are sustained by stable prior expectations each person has about others in their community, including about complete strangers \cite{lewis_convention:_1969, hawkins2019emergence}.
An English speaker may order an ``espresso'' at any café in the United States and expect to receive roughly the same drink.

At the same time, meaning is remarkably flexible and \emph{partner-specific}.
The same words may be interpreted differently by different listeners. 
Interactions between friends and colleagues are filled with proper names, technical jargon, slang, shorthand, and inside jokes, many of which are unintelligible to outside observers.
Words may even take on new \emph{ad hoc} senses over the course of a conversation \cite{clark_using_1996}.

The tension between these two basic observations has posed a challenging puzzle for theories of convention.
Influential computational accounts explaining how conventions emerge in populations \cite<e.g.>{skyrms2010signals,steels2011modeling,barr_establishing_2004,young_evolution_2015} typically do not allow for partner-specific meaning at all.
These accounts examine groups of interacting agents who update their representation of language after each interaction.
While the specific update rule ranges from simple associative mechanisms \cite<e.g.>{steels_self-organizing_1995} or heuristics \cite<e.g>{Young96_EconomicsOfConvention} to sophisticated deep reinforcement learning algorithms \cite{tieleman2019shaping,graesser2019emergent,mordatch2017emergence}, all of these accounts assume that agents update a single, monolithic representation of language to be used with every partner, and that agents do not (knowingly) interact repeatedly with the same partner.

Conversely, accounts emphasizing rapid alignment \cite{pickering2004toward} and partner-specific common ground \cite{ClarkWilkesGibbs86_ReferringCollaborative} typically do not provide mechanisms by which community-wide conventions may arise over longer timescales.
The philosopher Donald Davidson articulated one of the most radical of these accounts.
According to \citeA{davidson1984communication,davidson_nice_1986}, it is exclusively the ability to coordinate on \emph{partner-specific} meanings that is ultimately responsible for successful communication.
This line of argument led \citeA{davidson_nice_1986} to memorably conclude that ``there is no such thing as a language'' (p.~265).\footnote{ In Davidson's terminology, the system of meanings derived from conventional expectations is called the agent's \emph{prior} theory, while the ad hoc systems they form are called their \emph{passing} theories. We maintain this distinction, but instead call these \emph{global} conventions and \emph{local} or \emph{ad hoc} conventions, respectively.}.

In this paper, we propose a theory of convention formation that reconciles community-level consensus with partner-specific common ground in a unified cognitive model.
We begin by formalizing the computational problem facing agents who must communicate with one another in a variable and non-stationary world. 
We argue that three core cognitive mechanisms are needed for an agent to solve this problem: 
\begin{enumerate}
\item \textbf{uncertainty} about what words mean to different partners
\item rapid \textbf{adaptation} to particular partners
\item  inductive \textbf{generalization} to unseen partners
\end{enumerate}
%\begin{enumerate}
%\item \textbf{Lexical uncertainty:} When we first encounter a new communication partner in a new context, we call upon some representation about what we think different signals mean to them. This representation of meaning must be sensitive to the overall statistics of the population: more people are familiar with the use of \emph{dog} to refer to the beloved pet than \emph{sclerotic aorta} to refer to the potentially dangerous health condition. It must also be sensitive to the immediate context of the interaction: a cardiologist should have different expectations about a novel colleague than a novel patient.
%\item \textbf{Rapid adaptation:} Within a few minutes of conversation, we can considerably strengthen our expectations about our partner's lexicon based on earlier utterances and feedback, and adjust our own usage accordingly. For example, even if we are not initially familiar with the term \emph{sclerotic aorta}, a few minutes spent discussing the condition in simpler terms should make us more confident using the term with that partner in the future. This social learning mechanism must allow for signal \emph{reduction} -- simpler, more efficient ways of referring to the same thing over time -- and \emph{path-dependence}: early reinforcement of certain meanings increases their later usage, however arbitrary or provisional they began. 
%\item \textbf{Generalization:} When we encounter the same partner in a new context, we should expect some `stickiness' from previous learning. Language does not reset at context boundaries. In addition, the lexical model we've learned within a conversation should be largely \emph{partner-specific}. Just because we now expect Partner A to be familiar with a \emph{sclerotic aorta} shouldn't radically change our expectations about Partner B. Over enough interactions with different language users, however, our initial representations should be able to shift to take these data into account. To generalize appropriately, we must be able to correctly attribute whether a usage is idiosyncratic to a particular speaker, or a global convention we should expect to hold across the whole community.
%\end{enumerate}
Finally, we show that a hierarchical Bayesian model satisfies these desiderata and successfully explains three key phenomena in the empirical literature that have proved evasive for previous accounts: (P1) gradual reduction to simpler and more efficient referring expressions in dyadic interaction with a single partner, (P2) context-sensitivity in the content of which conventions form, and (P3) the gradient of partner-specificity and generalization in the emergence of conventions in small networks.

\rdh{Signpost rest of intro here: In the remainder of this section, we introduce... }

\subsection{Evidence from repeated reference games}

These phenomena have largely been explored using the \emph{repeated reference game} paradigm...

\rdh{Introduce paradigm here, and briefly mention these three key phenomena, enough for }