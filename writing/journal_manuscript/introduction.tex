
% 4 paragraph intro

To communicate successfully, speakers and listeners must share a common system of semantic meaning in the language they are using. 
These meanings are \emph{conventional} in the sense that they are sustained by stable prior expectations each person has about others in their community, including complete strangers \cite{lewis_convention:_1969, hawkins2019emergence}.
An English speaker may order an ``espresso'' at any café in the United States and expect to receive roughly the same drink.

At the same time, meaning is remarkably flexible and \emph{partner-specific}.
The same words may be interpreted differently by different listeners, and may take on new \emph{ad hoc} senses over the course of a conversation \cite{clark_using_1996}. 
Interactions between friends and colleagues are filled with proper names, technical jargon, slang, shorthand, and inside jokes, many of which are unintelligible to outside observers.

The tension between these two basic observations has posed a challenging puzzle for theories of convention.
Computational accounts seeking to explain how conventions emerge do not typically allow for partner-specific meaning at all: agents are assumed to update a monolithic representation of language \cite{steels_self-organizing_1995,barr_establishing_2004,young_evolution_2015}.
Conversely, accounts emphasizing coordination on partner-specific meaning via common ground typically do not provide a mechanism by which community-wide conventions may form.
According to the philosopher Donald Davidson, for example, it is the ability to coordinate on \emph{partner-specific} meanings that is ultimately responsible for successful communication, since no two individuals will ever share exactly the same conventional meanings \cite{davidson1984communication,davidson_nice_1986}\footnote{This line of argument led Davidson to memorably claim that ``there is no such thing as a language.'' In his terminology, the system of meanings derived from conventional expectations is called the agent's \emph{prior} theory, while the ad hoc systems they form against the backdrop of interaction with a particular partner is called their \emph{passing} theory. We maintain this important distinction, but will instead call these \emph{global} conventions and \emph{local} or \emph{ad hoc} conventions, consistent with our claim that prior theories and passing theories belong to the same unified structure.}.

In this paper, we propose a theory of convention formation that reconciles community-level consensus with partner-specific common ground through the process of meta-learning.
We begin by formalizing the computational problem facing agents who must communicate with one another in a variable and non-stationary world. 
We argue that three core cognitive mechanisms are needed for an agent to solve this problem: 
\begin{enumerate}
\item \textbf{uncertainty} about what words mean to different partners
\item rapid \textbf{adaptation} to particular partners, in context
\item  inductive \textbf{generalization} to unseen partners
\end{enumerate}
%\begin{enumerate}
%\item \textbf{Lexical uncertainty:} When we first encounter a new communication partner in a new context, we call upon some representation about what we think different signals mean to them. This representation of meaning must be sensitive to the overall statistics of the population: more people are familiar with the use of \emph{dog} to refer to the beloved pet than \emph{sclerotic aorta} to refer to the potentially dangerous health condition. It must also be sensitive to the immediate context of the interaction: a cardiologist should have different expectations about a novel colleague than a novel patient.
%\item \textbf{Rapid adaptation:} Within a few minutes of conversation, we can considerably strengthen our expectations about our partner's lexicon based on earlier utterances and feedback, and adjust our own usage accordingly. For example, even if we are not initially familiar with the term \emph{sclerotic aorta}, a few minutes spent discussing the condition in simpler terms should make us more confident using the term with that partner in the future. This social learning mechanism must allow for signal \emph{reduction} -- simpler, more efficient ways of referring to the same thing over time -- and \emph{path-dependence}: early reinforcement of certain meanings increases their later usage, however arbitrary or provisional they began. 
%\item \textbf{Generalization:} When we encounter the same partner in a new context, we should expect some `stickiness' from previous learning. Language does not reset at context boundaries. In addition, the lexical model we've learned within a conversation should be largely \emph{partner-specific}. Just because we now expect Partner A to be familiar with a \emph{sclerotic aorta} shouldn't radically change our expectations about Partner B. Over enough interactions with different language users, however, our initial representations should be able to shift to take these data into account. To generalize appropriately, we must be able to correctly attribute whether a usage is idiosyncratic to a particular speaker, or a global convention we should expect to hold across the whole community.
%\end{enumerate}
We show that a hierarchical Bayesian model satisfies these desiderata and successfully explains three key phenomena in the empirical literature that have proved evasive for previous accounts: (P1) gradual reduction to simpler and more efficient referring expressions in dyadic interaction with a single partner, (P2) context-sensitivity in the content of which conventions form, and (P3) the gradient of partner-specificity and generalization in the emergence of conventions in small networks.

\subsection{Theoretical approaches to adaptation and convention}

\todo[inline]{TODO: I decided to save the empirical background for each phenomenon for the specific section intros... I want to keep this bit short but wasn't sure how short before getting to our model? Maybe the intro is also the place to put some general background about repeated reference games as a task paradigm, even if I don't need to go into much detail about each of the three empirical phenomena yet?}

\paragraph{Priming-based theories of dyadic alignment}

According to the \emph{interactive alignment} account \cite{pickering2004toward,pickering2006alignment, garrod2009joint}, coordination on higher-level mental representations proceeds primarily through automatic and egocentric priming of surface statistics at lower levels. 
By appealing to classic spreading-activation connectionist models \cite<e.g.>{roelofs1992spreading}, \citeA{pickering2004toward} have argued that activating phonetic or syntactic features that are associated with specific lemmas in the lexicon can percolate to strengthen higher semantic levels of representation \cite{pickering1998representation}.
Thus, unconsciously coupling word choices \cite{louwerse2012behavior}, syntax \cite{gruberg2019syntactic,levelt1982surface}, body postures \cite{lakin2003using}, speech rate \cite{giles1991contexts}, or even informational complexity \cite{abney2014complexity} in dialogue could potentially contribute to the coordination of higher-level semantic representations.
%It does not make predictions about partner-specificity, since in principle priming is a non-social accessibility-based mechanism.

\paragraph{Agent-based theories of consensus}

Agent-based models use simple update rules derived from reinforcement learning to update a monolithic model after each interaction \cite<see>{skyrms2010signals}.
\citeA{spike_minimal_2017} distill three common features of these accounts: (1) creation and transmission of referential information, (2) a systemic bias against ambiguity, and (3) some form of information loss.
Critically, for our purposes, these models do not explicitly represent uncertainty about the meanings used by partners and do not represent the identity of those partners.
