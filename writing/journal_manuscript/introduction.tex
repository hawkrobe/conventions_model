
% 4 paragraph intro

To communicate successfully, speakers and listeners must share a common system of semantic meaning in the language they are using. 
These meanings are \emph{social conventions} in the sense that they are arbitrary to some degree, but sustained by stable prior expectations each person has about others in their community \cite{lewis_convention:_1969,bicchieri_grammar_2006, hawkins2019emergence}.
These expectations extend to complete strangers.
For example, an English speaker may order an ``espresso'' at any café in the United States and expect to receive roughly the same drink.

At the same time, meaning is remarkably flexible and \emph{partner-specific}.
The same words may be interpreted differently by different listeners. 
Interactions between friends and colleagues are filled with proper names, technical jargon, slang, shorthand, and inside jokes, many of which are unintelligible to outside observers.
Words may even take on new \emph{ad hoc} senses over the course of a conversation \cite{clark_using_1996}.

The tension between these two basic observations has posed a challenging puzzle for theories of convention.
Influential computational accounts explaining how conventions emerge in populations \cite<e.g.>{skyrms2010signals,steels2011modeling,barr_establishing_2004,young_evolution_2015} typically do not allow for partner-specific meaning at all.
These accounts examine groups of interacting agents who update their representation of language after each interaction.
While the specific update rule ranges from simple associative mechanisms \cite<e.g.>{steels_self-organizing_1995} or heuristics \cite<e.g>{Young96_EconomicsOfConvention} to sophisticated deep reinforcement learning algorithms \cite{tieleman2019shaping,graesser2019emergent,mordatch2017emergence}, all of these accounts assume that agents update a single, monolithic representation of language to be used with every partner, and that agents do not (knowingly) interact repeatedly with the same partner.

Conversely, accounts emphasizing rapid alignment \cite{pickering2004toward} and partner-specific common ground \cite{ClarkWilkesGibbs86_ReferringCollaborative} typically do not provide mechanisms by which community-wide conventions may arise over longer timescales.
The philosopher Donald Davidson articulated one of the most radical of these accounts.
According to \citeA{davidson1984communication,davidson_nice_1986}, it is exclusively the ability to coordinate on \emph{partner-specific} meanings that is ultimately responsible for successful communication.
This line of argument led \citeA{davidson_nice_1986} to memorably conclude that ``there is no such thing as a language'' (p.~265), and to abandon appeals to convention altogether\footnote{ In Davidson's terminology, the system of meanings derived from conventions is called the agent's \emph{prior} theory, while the ad hoc systems they form are called their \emph{passing} theories. We maintain this distinction, but instead call these \emph{global} conventions and \emph{local} or \emph{ad hoc} conventions, respectively, to emphasize their commonality.}.

In this paper, we propose a theory of coordination and convention that reconciles community-level consensus with partner-specific common ground in a unified cognitive model.
We begin by formalizing the computational problem facing agents who must communicate with one another in a variable and non-stationary world. 
We argue that three core cognitive mechanisms are needed for an agent to solve this problem: 
\begin{description}
\item[M1:] \textbf{uncertainty} about what words mean to different partners
\item[M2:] rapid \textbf{learning} to adapt to particular partners
\item[M3:] inductive \textbf{generalization} to unseen partners
\end{description}
%\begin{enumerate}
%\item \textbf{Lexical uncertainty:} When we first encounter a new communication partner in a new context, we call upon some representation about what we think different signals mean to them. This representation of meaning must be sensitive to the overall statistics of the population: more people are familiar with the use of \emph{dog} to refer to the beloved pet than \emph{sclerotic aorta} to refer to the potentially dangerous health condition. It must also be sensitive to the immediate context of the interaction: a cardiologist should have different expectations about a novel colleague than a novel patient.
%\item \textbf{Rapid adaptation:} Within a few minutes of conversation, we can considerably strengthen our expectations about our partner's lexicon based on earlier utterances and feedback, and adjust our own usage accordingly. For example, even if we are not initially familiar with the term \emph{sclerotic aorta}, a few minutes spent discussing the condition in simpler terms should make us more confident using the term with that partner in the future. This social learning mechanism must allow for signal \emph{reduction} -- simpler, more efficient ways of referring to the same thing over time -- and \emph{path-dependence}: early reinforcement of certain meanings increases their later usage, however arbitrary or provisional they began. 
%\item \textbf{Generalization:} When we encounter the same partner in a new context, we should expect some `stickiness' from previous learning. Language does not reset at context boundaries. In addition, the lexical model we've learned within a conversation should be largely \emph{partner-specific}. Just because we now expect Partner A to be familiar with a \emph{sclerotic aorta} shouldn't radically change our expectations about Partner B. Over enough interactions with different language users, however, our initial representations should be able to shift to take these data into account. To generalize appropriately, we must be able to correctly attribute whether a usage is idiosyncratic to a particular speaker, or a global convention we should expect to hold across the whole community.
%\end{enumerate}
Finally, we present a hierarchical Bayesian model that satisfies these desiderata and demonstrate that it successfully explains three key phenomena in the empirical literature that have proved evasive for previous accounts: 
\begin{description}
\item[P1:] convergence to more \textbf{efficient} referring expressions with a single partner, 
\item[P2:] \textbf{context-sensitivity} in which labels become conventionalized
\item[P3:] the gradient of \textbf{partner-specificity} as conventions emerge in small networks.
\end{description}
In the remainder of this section, we introduce the \emph{repeated reference game} paradigm at the center of this literature, review the empirical evidence for each of these phenomena, and briefly discuss some limitations of previous accounts.

\subsection{Empirical evidence from repeated reference games}

A core function of language is \emph{reference}: using words to convey the identity of an object in the environment. 
Following \citeA{wittgenstein2009philosophical}, empirical studies of coordination and convention in communication have predominantly focused on the subset of language use captured by simple ``reference games.'' 
In a reference game task, participants are assigned to speaker and listener roles and shown a context $\mathcal{C}$ of possible referential targets (e.g. images).
On each trial, the speaker is asked to produce a referring expression $u$ --- typically a noun phrase --- that will allow the listener to select the intended target object $o$ from the distractors in context.
Critically, a \emph{repeated reference game} asks them to refer to the same targets multiple times as they build up a shared history of interaction, or \emph{common ground,} with their partner (see Table \ref{table:parameters} in Appendix for a review of the different axes along which the design may varied).

Unlike general studies of referring expression generation, studies of coordination and convention use novel, ambiguous stimuli that participants do not already have strong conventions for.
Unlike agent-based simulations of convention formation on large networks, which typically match agents with a new partner for each trial, repeated reference games ensure that participants maintain the same partner for the duration of the game.
This design allows us to observe how the speaker's referring expressions for the same objects change as a function of interaction with a partner.

\paragraph{Increasing efficiency}

The most well-known phenomenon observed in repeated reference games is a dramatic reduction in message length over multiple rounds \cite{krauss_changes_1964, ClarkWilkesGibbs86_ReferringCollaborative, hawkins2020characterizing}. 
The first time participants refer to a figure, they tend to use a lengthy, detailed description (``the upside-down martini glass in a wire stand'') but with a small number of repetitions -- between 3 to 6, depending on the pair of participants -- the description may be cut down to the limit of just one or two words (``martini''). 
These final messages are as short or shorter than the messages participants choose to allow \emph{themselves} to choose the target in the future  \cite{FussellKrauss89_IntendedAudienceCommonGround} and are often incomprehensible to overhearers who were not present for the initial messages \cite{SchoberClark89_Overhearers}.
These observations set up our first puzzle of \emph{ad hoc} convention formation in dyads: how does a short word or phrase that would have been completely ineffective for communicating under the global conventions of a language become perfectly understandable over mere minutes of interaction? 

\paragraph{Context-sensitivity}

While a degree of arbitrariness is central to conventionality -- there must exist more than one solution that would work equally well -- this does not necessarily imply that all possible conventions for a meaning are equally likely in practice, or even that all meanings are equally likely to become conventionalized in the first place \cite{HawkinsGoldstone16_SocialConventions}.
Functional accounts of language have frequently observed that lexical systems are well-calibrated to the statistics of the environment \cite{gibson2019efficiency}.
This ``optimal expressivity'' hypothesis has accounted well for the lexical distributions found in natural languages across semantic domains like color words and kinship categories \cite{KempRegier12_KinshipCategories,regier201511,gibson2017color,kemp2018semantic}.
For example, languages in warm regions ought to be more likely to collapse the distinction between ice and snow into a single word, simply because there are fewer occasions that require distinguishing between the two \cite{regier2016languages}. 

Context-sensitivity has also been investigated in the lab using repeated reference games.
Both the distribution of conventions that emerge when communicating with artificial languages \cite{WintersKirbySmith14_LanguagesAdapt, KirbyTamarizCornishSmith15_CompressionCommunication} and the efficient \emph{ad hoc} labels that dyads converge to when using natural language \cite{hawkins2020characterizing} have been found to depend on the distractors in the context $\mathcal{C}$.
Thus, while there is abundant empirical evidence for context-sensitivity in the \emph{outcomes} that result from convention formation processes, it is currently unclear what cognitive mechanisms are necessary to give rise to such outcomes.
That is, context-sensitivity has not yet been grounded in a cognitive and mechanistic account of the processes unfolding in the minds of individual agents as they interact.

\paragraph{Partner-specificity}

\rdh{Introduce paradigm here, and briefly mention these three key phenomena, enough for }